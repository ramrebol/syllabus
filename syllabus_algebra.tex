% Created 2022-12-26 lun 22:48
% Intended LaTeX compiler: pdflatex
\documentclass[11pt]{article}
\usepackage[utf8]{inputenc}
\usepackage[T1]{fontenc}
\usepackage{graphicx}
\usepackage{grffile}
\usepackage{longtable}
\usepackage{wrapfig}
\usepackage{rotating}
\usepackage[normalem]{ulem}
\usepackage{amsmath}
\usepackage{textcomp}
\usepackage{amssymb}
\usepackage{capt-of}
\usepackage{hyperref}
\usepackage{parskip}
\usepackage[letterpaper,margin=15mm]{geometry}
\usepackage{longtable}
\usepackage{titling}
\setlength{\droptitle}{-60px}
\date{}
\title{Syllabus}
\hypersetup{
 pdfauthor={Ramiro James Rebolledo Cormack},
 pdftitle={Syllabus},
 pdfkeywords={},
 pdfsubject={},
 pdfcreator={Emacs 27.1 (Org mode 9.3)}, 
 pdflang={English}}
\begin{document}

\maketitle

\vspace{-60px}

\section*{Programa de Asignatura}
\label{sec:org750cca4}

\section{Información básica}
\label{sec:orgfa3fb76}
Unidad académica responsable: Facultad de Ingeniería Agrícola, Departamento de Agroindustrias.

Carrera: Ingeniería Civil Agrícola

Ámbitos de desempeño: Recursos naturales, gestión de proyectos, innovación y desarrollo tecnológico.

\section{Identificación}
\label{sec:org5f1375a}

\begin{center}
\begin{tabular}{|l|l|l|l|}
\hline
\multicolumn{4}{|l|}{Nombre de la asignatura: Álgebra y Trigonometría} \\
\hline
Código: 235300 & Crédito UdeC: 5 & Créditos SCT: 8 & \\
\hline
\multicolumn{4}{|l|}{Prerrequisitos: No tiene} \\
\hline
Modalidad: Presencial & Calidad: Obligatorio & Duración: Semestral & \\
\hline
\multicolumn{4}{|l|}{Semestre en el plan de estudio: Ingeniería Civil Agrícola 3006-2019-Primer semestre} \\
\hline
\multicolumn{4}{|l|}{Nombre de los cursos asociados a la asignatura: No es asignatura integradora} \\
\hline
\multicolumn{4}{|l|}{Trabajo académico: 13 horas} \\
\hline
Horas teóricas: 4 & Horas prácticas: 2 & Horas de laboratorio: 0 & Horas de trabajo autónomo \\
 & & & de cada estudiante: 6 \\
\hline
\end{tabular}
\end{center}

\section{Descripción}
\label{sec:org4c6b061}
Asignatura teórico-práctica que introduce al estudiante en los conceptos básicos del Álgebra Elemental y de la Trigonometría, contribuyendo a desarrollar la capacidad de abstracción y análisis, esenciales para toda ingeniería.

\section{Competencias}
\label{sec:org9751400}
Esta asignatura contribuye al desarrollo de las siguientes competencias del perfil de egreso para los estudiantes de Ingeniería Civil Agrícola:

\begin{enumerate}
\item Concebir, diseñar, implementar y ejecutar soluciones de ingeniería asociadas a la agricultura, considerando las dimensiones sociales, éticas, económicas, tecnológicas y ambientales.

\item Diseñar, planificar y evaluar obras hidráulicas, considerando criterios de seguridad, adecuación y confiabilidad para los usuarios, comprendiendo el ejercicio de su profesión como una oportunidad de servir y aportar constructivamente a la sociedad.

\item Diseñar y evaluar sistemas de riego y drenaje intra y extra predial para la optimización y sustentabilidad del recurso hídrico, considerando la demanda de los cultivos y los equipos e implementos necesarios para su correcta operación.

\item Diseñar y evaluar proyectos asociados al uso de energías renovables no convencionales, considerando las dimensiones sociales, éticas, económicas, tecnológicas y ambientales.

\item Desarrollar y evaluar proyectos de ingeniería para optimizar recursos tecnológicos, humanos, materiales y financieros, conformando equipos interdisciplinarios enfocados en el logro de los objetivos propuestos.

\item Desarrollar e implementar tecnologías para generar soluciones innovadoras en el sector agrícola, visualizando oportunidades y desafíos en los que emprender.

\item Analizar la coherencia de los juicios propios y ajenos, y valorar las implicaciones personales y sociales de los mismos, dentro del contexto de su desempeño profesional.

\item Comunicar aspectos técnicos de ingeniería, de manera efectiva y haciendo uso correcto del lenguaje, aplicando recursos tecnológicos y habilidades sociales.
\end{enumerate}


\section{Resultados de Aprendizaje}
\label{sec:org91621e8}

Al completar en forma exitosa esta asignatura, los estudiantes serán capaces de:

\begin{enumerate}
\item Aplicar el concepto de funciones y sus características distintivas en el análisis de problemas de Ingeniería.

\item Utilizar el razonamiento lógico para argumentar y comunicar información científica.

\item Aplicar las diferentes propiedades del conjunto de los números complejos para expresar conceptos y soluciones de problemas simples en el área de la ingeniería.

\item Plantear y resolver problemas que involucren conceptos de trigonometría y resolución de ecuaciones (lineales, cuadráticas, trigonométricas, logarítmicas, exponenciales y polinomiales) en distintas áreas de las ciencias de la ingeniería.
\end{enumerate}

\section{Contenidos}
\label{sec:org3c37485}

\begin{itemize}
\item \textbf{Lógica y Conjuntos:} Proposición. Conectivos Lógicos. Tablas de verdad. Tautología. Contradicción y Contingencia. Proposiciones lógicamente Equivalentes. Implicación Lógica. Teorema. Teorema recíproco, contrario y Contra-recíproco. Métodos de demostración: Directo e Indirecto (Reducción al absurdo). Noción y notación de conjunto. Relación de Pertenencia y de Inclusión de conjuntos. Conjunto vacío. Conjunto de partes. Operaciones con conjuntos: Conjunto Universo, diferencia, complemento, unión e intersección de conjuntos. Conjuntos disjuntos. Propiedades. Producto Cartesiano. Partición de conjunto.  Función Proposicional y Conjunto de Validez. Cuantificadores Lógicos. Negación de proposiciones con cuantificadores lógicos.

\item \textbf{Funciones:} Definición de Relación y Función. Dominio y Recorrido. Igualdad de funciones. Funciones Inyectivas, Sobreyectivas y Biyectivas. Funciones restringidas. Función real de variable real. Gráfica de funciones. Álgebra de funciones: adición, sustracción, producto y cuociente de funciones. Composición de funciones. Función Inversa; Propiedades. Funciones por tramos.

\item \textbf{Funciones Circulares:} Definiciones y Representación Gráfica. Identidades fundamentales. Funciones Trigonométricas de Sumas, Diferencias y Múltiplos. Teoremas del Seno y del Coseno. Identidades y Ecuaciones Trigonométricas. Funciones Trigonométricas Inversas y sus gráficas. Aplicaciones.

\item \textbf{Números Complejos:} Plano Complejo. Propiedades y operatoria. Forma Polar y Exponencial. Teorema de De Moivre. Raíces de un Complejo. Resolución de ecuaciones.

\item \textbf{Polinomios:} Polinomios y Funciones Polinomiales. Álgebra de Polinomios. Raíces y factorización de un Polinomio. Teorema Fundamental del Álgebra. Descomposición en Suma de Fracciones Parciales.

\item \textbf{Inducción Matemática y Teorema del Binomio:} Principio de Inducción Matemática. Coeficientes Binomiales y Teorema del Binomio.
\end{itemize}


\section{Metodología}
\label{sec:org0c05697}

Clases teóricas y prácticas. Discusión de material escrito y guías de ejercicios orientados a problemas de ingeniería basados en la bibliografía del curso. Las actividades anteriores se reforzarán con atención grupal con el ayudante para resolver ejercicios y dudas (una hora semanal), y atención individual en oficina del profesor para resolver dudas.


\section{Evaluación}
\label{sec:orgc3cadd0}
Se evaluará de acuerdo al Reglamento de Docencia de Pregrado de la Facultad de Ingeniería Agrícola. Se realizarán 3 evaluaciones (E1, E2 y E3) de 20\%, 30\% y 30\%, respectivamente, en las fechas indicadas en la planificación. Además se harán tests y tareas cuyo promedio (T) será el 20\% y que serán avisados con una semana de anticipación. Los tests serán rendidos en clases y las tareas serán desarrolladas fuera del horario de clases.
Con respecto a las tareas, la primera tarea debe entregarse en una carpeta simple con archivador con el nombre, carrera y número de matrícula en un lugar visible de la carpeta. Cada tarea debe tener esta misma información (nombre, carrera y número de matrícula) en la parte superior de la primera hoja además del número de la tarea y fecha. Cada tarea debe entregarse en hojas blancas, y debe contener tanto el enunciado como la respuesta de TODAS las preguntas. Las tareas estarán disponibles en Canvas una semana antes de la fecha de entrega. Cada día de retraso en la entrega corresponderá a un descuento de 0.5 puntos en su calificación. Parte de la calificación de la tarea será, además de la buena redacción, la presentación de la tarea, es decir, respuestas mal redactadas, manchas en la hojas, desorden o arrugas en las hojas significarán un descuento en la nota de la tarea. Las tareas serán guardadas por el profesor. El día de entrega o revisión de la tarea el profesor puede hacer preguntas a cualquier alumno sobre lo respondido en la tarea, y si si no se demuestra dominio en lo respondido se puede descontar puntos. En el caso de tareas grupales, si uno de los miembros del grupo no demuestra comprensión de lo respondido en la tarea, el descuento en la nota será a todo el grupo.

Los alumnos que obtengan una calificación menor que 3.95 pueden rendir la Evaluación de Recuperación (ER) la cual tiene carácter de examen con una ponderación de 40\% de la nota final.
Si el alumno no rinde la ER, la nota final (NF) se calcula como sigue:
\begin{center}
NF  = 0.2*E1 + 0.3*E2 + 0.3*E3 + 0.2*T
\end{center}

En el caso de que NF sea mayor o igual que 3.95 el alumno está aprobado con nota NF. En caso contrario el alumno puede rendir la ER calculándose la NF como sigue

\begin{center}
NF  = 0.6*(0.2*E1 + 0.3*E2 + 0.3*E3 + 0.2*T) + 0.4*ER
\end{center}

Quien no rinda E1, E2 o E3 pero justifique adecuadamente su inasistencia presentando un certificado médico ante la DISE o sea autorizado por su jefe de carrera, podrá rendir la evaluación faltante al final del semestre en una fecha a convenir (después de E3 y antes de ER). La no rendición de E1, E2 o E3 es calificada con concepto NCR. La no rendición de un test o tarea será calificado con nota mínima.

\section{Evidencias}
\label{sec:orgecb0cab}

Al no ser asignatura integradora, este ítem no procede.

\section{Bibliografía}
\label{sec:orga1a976f}

\textbf{Bibliografía Básica}

\begin{enumerate}
\item Devaud, G. et. al. (1996). Álgebra. Universidad de Concepción. s/ISBN

\item Zill D. y Dewar J. (2012). Álgebra, Trigonometría y Geometría Analítica. McGraw-Hill. ISBN: 9786071507143.
\end{enumerate}

\textbf{Bibliografía Complementaria}

\begin{enumerate}
\item Sandoval J., Flores A. y Gálvez M. (2010). Guías de Práctico Álgebra y Trigonometría. Universidad de Concepción. ISBN: 9789568029883.

\item Devaud, G. et. al. (1996). Ejercicios propuestos y resueltos de Álgebra y Álgebra lineal. Universidad de Concepción. s/ISBN
\end{enumerate}

Nota: Todos estos libros están disponibles en biblioteca de manera física.

\section{Lineamientos Institucionales}
\label{sec:orgbc03701}

\begin{itemize}
\item Universidad interdisciplinaria de impacto nacional con proyección internacional
\item Innovación para la excelencia
\item Comunidad comprometida con la inclusión y la equidad de género
\item Desarrollo sustentable de la institución.
\end{itemize}

\section{Planificación}
\label{sec:orgff9c800}


{\scriptsize
\begin{center}
\begin{longtable}{|l|l|l|l|l|l|l|l|}
\hline
Semana & Competencia & Resultado & Contenidos & Actividad de & Evaluación & Responsable & Horas de \\
 & & de & & aprendizaje & & & trabajo \\
 & & aprendizaje & & & & & académico \\
 & & & & & & & \\
 & & & & & & & \\
\hline
S1:16/8 & & 2 & Presentación del syllabus. & Primera parte & Formativa & Docente y & 2 \\
 & & & Módulo 1: Lógica: Proposiciones & teórica. Finaliza con & & alumnos & \\
 & & & y tablas de verdad. Conectivos & problemas que alumnos & & & \\
 & & & lógicos, tautología, & deben resolver usando & & & \\
 & & & contradicción y contingencia. & tablas de verdad. & & & \\
 & & & & & & & \\
\hline
18/8 & & 2 & Funciones proposicionales y & Primera parte & Formativa & Docentes y & 2 \\
 & & & cuantificadores. Implicancias y & teórica. Segunda & & alumnos & \\
 & & & equivalencias. & parte ejercicios. & & & \\
\hline
S2:22/8 & & 2 & Ejercicios. & Los estudiantes & Formativa & Docente y & 2 \\
 & & & & resuelven problemas & & alumnos & \\
 & & & & del Listado 1 & & & \\
 & & & & mientras el profesor & & & \\
 & & & & resuelve dudas. & & & \\
 & & & & & & & \\
\hline
23/8 & & 2 & Tarea 1. Teoremas y métodos de & Primera parte & Formativa & Docente y & 2 \\
 & & & demostración. & teórica. Segunda & & alumnos & \\
 & & & & parte ejercicios en & & & \\
 & & & & grupo. & & & \\
\hline
25/8 & & 1, 2 & Módulo 2: Teoría de conjuntos: & Primera parte & Formativa & Docente y & 2 \\
 & & & Definiciones básicas, noción de & teórica. Segunda & & alumnos & \\
 & & & pertenencia, subconjunto e & parte ejercicios & & & \\
 & & & igualdad. Conjunto universo y & grupales. & & & \\
 & & & vacío. & & & & \\
\hline
S3:29/8 & & 1, 2 & Propiedades de conjuntos & Primera parte & Formativa & Docente y & 2 \\
 & & & (demostraciones y ejemplos). & teórica. Segunda & & alumnos & \\
 & & & & parte ejercicios & & & \\
 & & & & grupales. & & & \\
\hline
30/8 & & 1, 2 & Revisión Tarea 1: Demostración & Profesor explica la & Formativa & Docente y & 2 \\
 & & & de conjuntos. & pauta de la Tarea 1 y & & alumnos & \\
 & & & & discute dudas. Luego & & & \\
 & & & & los alumnos resuelven & & & \\
 & & & & problemas del Listado & & & \\
 & & & & 2 (conjuntos) & & & \\
 & & & & supervisados por el & & & \\
 & & & & profesor. & & & \\
 & & & & & & & \\
 & & & & & & & \\
\hline
1/9 & & 1, 2 & Tarea 2. Ejercicios de & Primera parte & Formativa & Docente y & 2 \\
 & & & demostraciones de & teórica. Segunda & & alumnos & \\
 & & & conjuntos. Cardinalidad y & parte ejercicios & & & \\
 & & & diagrama de Venn. & grupales. & & & \\
\hline
S3: 5/9 & & 2 & Módulo 3: Símbolo de sumatoria. & Primera parte & Formativa & Docente y & 2 \\
 & & & Principio y Teorema de & teórica. Segunda & & alumnos & \\
 & & & inducción. & parte ejercicios & & & \\
 & & & & grupales. & & & \\
\hline
5/9 & & 2 & Ayudantía & Los alumnos & Formativa & Ayudante y & 1 \\
 & & & & resuelven ejercicios & & alumnos & \\
 & & & & de sumatoria & & & \\
 & & & & (Listado 3) & & & \\
 & & & & supervisados por el & & & \\
 & & & & ayudante. & & & \\
\hline
6/9 & & 2 & Ejercicios de inducción, & Los alumnos & Formativa & Docente y & 2 \\
 & & & demostración de propiedades de & resuelven ejercicios & & alumnos & \\
 & & & Progresión Geométrica y & de inducción & & & \\
 & & & Aritmética. & (listado 3) & & & \\
 & & & & supervisados por el & & & \\
 & & & & profesor. & & & \\
\hline
8/9 & & 2 & Tarea 3. Ejercicios módulo 3 & Alumnos resuelven & Formativa & Docente y & 2 \\
 & & & & ejercicios del & & alumnos & \\
 & & & & Listado 3 & & & \\
 & & & & supervisados por el & & & \\
 & & & & profesor. & & & \\
\hline
12/9 & & 2 & Ayudantía. & Alumnos resuelven & Formativa & Ayudante y & 1 \\
 & & & & ejercicios tipo & & alumnos & \\
 & & & & certamen como & & & \\
 & & & & preparación para el & & & \\
 & & & & Certamen 1 con la & & & \\
 & & & & ayuda de ayudante. & & & \\
\hline
13/9 & & 2 & Revisión Tarea 3. Ejercicios de & Se analiza la & Formativa & Docente y & 2 \\
 & & & preparación Certamen 1. & solución (pauta) de & & alumnos & \\
 & & & & la Tarea 3, y se & & & \\
 & & & & resuelven problemas y & & & \\
 & & & & dudas relacionadas & & & \\
 & & & & con el certamen 1. & & & \\
 & & & & & & & \\
\hline
15/6 & & & Certamen 1. & Evaluación escrita. & Sumativa & Profesor y & 2 \\
 & & & & & & alumnos & \\
\hline
S5:19/9 & \multicolumn{7}{l|}{Feriado} \\
\hline
20/9 & & 2 & Revisión Pauta Certamen & Se explica la Pauta & Formativa & Profesor y & 2 \\
 & & & 1. Recuerdo de conceptos básicos & del Certamen 1 & & alumnos & \\
 & & & de inecuaciones. & resolviendo & & & \\
 & & & & dudas. Se repasa & & & \\
 & & & & inecuaciones (vistas & & & \\
 & & & & en curso en & & & \\
 & & & & paralelo) mediante & & & \\
 & & & & ejemplos y & & & \\
 & & & & ejercicios en grupo. & & & \\
\hline
22/9 & & 1, 2 & Módulo 4: Funciones: Definición & Primera parte & Formativa & Profesor y & 2 \\
 & & & de Relación y Función. Funciones & teórica. Segunda & & alumnos & \\
 & & & reales. Dominio y recorrido. & parte ejercicios en & & & \\
 & & & & grupo. & & & \\
\hline
S6:26/9 & & 1, 2 & Ejercicios. & Los alumnos & Formativa & Profesor y & 2 \\
 & & & & resuelven ejercicios & & alumnos & \\
 & & & & del Listado 4 & & & \\
 & & & & supervisados por el & & & \\
 & & & & profesor. & & & \\
 & & & & & & & \\
\hline
26/9 & & 1, 2 & Ayudantía. & Alumnos resuelven & Formativa & Ayudante y & 1 \\
 & & & & ejercicios del & & alumnos & \\
 & & & & Listado 4 & & & \\
 & & & & supervisados por el & & & \\
 & & & & ayudante. & & & \\
\hline
27/9 & & 1, 2 & Funciones inyectivas y & Primera parte & Formativa & Profesor y & 2 \\
 & & & sobreyectivas. Función inversa. & teórica. Segunda & & alumnos & \\
 & & & Restricciones. & parte ejercicios en & & & \\
 & & & & grupo. & & & \\
\hline
29/9 & & 1, 2 & Tarea 4. Álgebra de funciones & Primera parte & Formativa & Profesor y & 2 \\
 & & & reales. Restricciones. Funciones & teórica. Segunda & & alumnos & \\
 & & & pares, impares y monótonas. & parte ejercicios en & & & \\
 & & & & grupo. & & & \\
\hline
S7:3/10 & & 1, 2 & Ejercicios Módulo 4. & Los alumnos & Formativa & Profesor y & 2 \\
 & & & & resuelven ejercicios & & alumnos & \\
 & & & & del Listado 4 con la & & & \\
 & & & & supervisión del & & & \\
 & & & & profesor. & & & \\
\hline
3/10 & & 1, 2 & Ayudantía. & Los alumnos & Formativa & Ayudante y & 1 \\
 & & & & resuelven ejercicios & & alumnos & \\
 & & & & del Listado 4 con la & & & \\
 & & & & supervisión del & & & \\
 & & & & ayudante. & & & \\
\hline
4/10 & & 1, 2 & Revisión Tarea 4 y ejercicios. & Primera parte & Formativa & Profesor y & 2 \\
 & & & & análisis de la Pauta & & alumnos & \\
 & & & & de la Tarea 4. & & & \\
 & & & & Segunda parte & & & \\
 & & & & ejercicios en grupo. & & & \\
\hline
6/10 & & 1, 2 y 4 & Tarea 5, Módulo 4. Funciones & Primera parte & Formativa & Profesor y & 2 \\
 & & & exponencial y logaritmo. & teórica. Parte final & & alumnos & \\
 & & & & los estudiantes & & & \\
 & & & & resuelven problemas & & & \\
 & & & & en grupo. & & & \\
 & & & & & & & \\
\hline
S8:10/10 & \multicolumn{7}{l|}{Feriado} \\
\hline
11/10 & & 1, 2 y 4 & Módulo 5: Funciones & Primera parte & Formativa & Profesor y & 2 \\
 & & & trigonométricas. Ángulos y sus & teórica. Parte final & & alumnos & \\
 & & & medidas. Cálculo de funciones & los estudiantes & & & \\
 & & & circulares en algunos ángulos & resuelven ejercicios & & & \\
 & & & específicos. Longitud de arco. & en grupo sobre los & & & \\
 & & & & conceptos de & & & \\
 & & & & trigonometría vistos & & & \\
 & & & & en la clase. & & & \\
 & & & & & & & \\
\hline
13/10 & & 1, 2 y 4 & Definición de funciones & Primera parte teórica & Formativa & Profesor y & 2 \\
 & & & circulares, sus gráficas y & haciendo uso de & & alumnos & \\
 & & & propiedades fundamentales. & geogebra explicar & & & \\
 & & & & gráficas. Parte final & & & \\
 & & & & los estudiantes & & & \\
 & & & & resuelven un & & & \\
 & & & & ejercicio en grupo & & & \\
 & & & & sobre gráficas de & & & \\
 & & & & funciones & & & \\
 & & & & trigonométricas. & & & \\
 & & & & & & & \\
\hline
S9:17/10 & & 1, 2 y 4 & Ejercicios. & Los estudiantes & Formativa & Profesor y & 2 \\
 & & & & resuelven ejercicios & & alumnos & \\
 & & & & del Listado 5 con la & & & \\
 & & & & supervisión del & & & \\
 & & & & profesor. & & & \\
\hline
17/10 & & 1, 2 y 4 & Ayudantía. & Los estudiantes & Formativa & Ayudante y & 1 \\
 & & & & resuelven ejercicios & & alumnos & \\
 & & & & del Listado 5 con la & & & \\
 & & & & supervisión del & & & \\
 & & & & ayudante. & & & \\
\hline
18/10 & & 1, 2 y 4 & Identidades trigonométricas. & Primera parte & Formativa & Profesor y & 2 \\
 & & & & teórica. Parte final & & alumnos & \\
 & & & & los estudiantes & & & \\
 & & & & resuelven ejercicios & & & \\
 & & & & en grupos sobre & & & \\
 & & & & identidades & & & \\
 & & & & trigonométricas. & & & \\
 & & & & & & & \\
\hline
20/10 & & 1, 2 y 4 & Otras identidades: Funciones & Primera parte & Formativa & Profesor y & 2 \\
 & & & trigonométricas de la suma y & teórica. Segunda & & alumnos & \\
 & & & diferencia, de ángulos dobles y & partelos estudiantes & & & \\
 & & & medios. & resuelven ejercicios & & & \\
 & & & & en grupo sobre lo & & & \\
 & & & & visto en la clase. & & & \\
\hline
S10:24/10 & & 1, 2 y 4 & Reducción de productos en sumas & Primera parte & Formativa & Profesor y & 2 \\
 & & & y viceversa. Funciones & expositiva. Segunda & & alumnos & \\
 & & & circulares inversas. & parte los & & & \\
 & & & & estudiantes & & & \\
 & & & & resuelven ejercicios & & & \\
 & & & & en grupo sobre lo & & & \\
 & & & & visto en clase. & & & \\
\hline
24/10 & & 1, 2 y 4 & Ayudantía. & Alumnos resuelven & Formativa & Ayudante y & 1 \\
 & & & & ejercicios del & & alumnos & \\
 & & & & Listado 5 bajo la & & & \\
 & & & & supervisión del & & & \\
 & & & & ayudante. & & & \\
\hline
25/10 & & 1, 2 y 4 & Tarea 6. Ecuaciones & Primera parte & Formativa & Profesor y & 2 \\
 & & & trigonométricas. & teórica. Segunda & & alumnos & \\
 & & & & parte los estudiantes & & & \\
 & & & & resuelven ejercicios & & & \\
 & & & & en grupo sobre & & & \\
 & & & & ecuaciones & & & \\
 & & & & trigonométricas. & & & \\
 & & & & & & & \\
\hline
27/10 & & 1, 2 y 4 & Ejercicios. & Alumnos resuelven & Formativa & Profesor y & 2 \\
 & & & & ejercicios del Listado & & alumnos & \\
 & & & & 6 bajo la supervisión & & & \\
 & & & & del profesor. & & & \\
 & & & & & & & \\
\hline
1/11 & \multicolumn{7}{l|}{Feriado} \\
\hline
3/11 & & & Certamen 2. & & Sumativa & Profesor y & 2 \\
 & & & & & & alumnos & \\
\hline
S12:7/11 & & 1, 2 y 4 & Tarea 7. Análisis de la pauta & Análisis de la Pauta & Formativa & Profesor y & 2 \\
 & & & del Certamen 2. Módulo 6: & del Certamen 2. Sigue & & alumnos & \\
 & & & Números & teóría sobre números & & & \\
 & & & complejos. Definición. Plano & complejos. Termina con & & & \\
 & & & complejo. Representación & los estudiantes & & & \\
 & & & binomial y como par & resolviendo problemas & & & \\
 & & & ordenado. Propiedades de módulo & en grupo sobre números & & & \\
 & & & y conjugado. & complejos. & & & \\
 & & & & & & & \\
\hline
7/11 & & 1, 2 y 4 & Ayudantía. & Los alumnos & Formativa & Ayudante y & 1 \\
 & & & & resuelven ejercicios & & alumnos & \\
 & & & & del Listado 7 bajo & & & \\
 & & & & la supervisión del & & & \\
 & & & & ayudante. & & & \\
\hline
8/11 & & 2 y 3 & Operaciones aritmética con & Primera parte & Formativa & Profesor y & 2 \\
 & & & números complejos, incluyendo & teórica. Segunda parte & & alumnos & \\
 & & & potencias con exponente natural. & los estudiantes & & & \\
 & & & Teorema de Moivre. & resuelven ejercicios en & & & \\
 & & & & grupo sobre & & & \\
 & & & & representación & & & \\
 & & & & exp y log de complejos. & & & \\
 & & & & & & & \\
 & & & & & & & \\
\hline
10/11 & & 2 y 3 & Representación polar, & Primera parte & Formativa & Profesor y & 2 \\
 & & & exponenciales y raíces de & teórica. Segunda parte & & alumnos & \\
 & & & complejos. Producto y cociente & los estudiantes & & & \\
 & & & números complejos usando forma & resuelven ejercicios en & & & \\
 & & & polar. & grupo sobre raíces de & & & \\
 & & & & números complejos. & & & \\
 & & & & & & & \\
 & & & & & & & \\
\hline
S13:14/11 & & 2 y 3 & Raíces de la unidad. & Primera parte & Formativa & Profesor y & 2 \\
 & & & & expositiva. Parte & & alumnos & \\
 & & & & final los & & & \\
 & & & & estudiantes & & & \\
 & & & & resuelven un & & & \\
 & & & & ejercicio en grupo & & & \\
 & & & & sobre raíces de la & & & \\
 & & & & unidad. & & & \\
\hline
14/11 & & 2 y 3 & Ayudantía. & Alumnos resuelven & Formativa & Ayudante y & 1 \\
 & & & & ejercicios del & & alumnos & \\
 & & & & Listado 7 bajo la & & & \\
 & & & & supervisión del & & & \\
 & & & & ayudante. & & & \\
\hline
15/11 & & 2 y 3 & Tarea 8. Ejercicios. & Resolución de & Formativa & Profesor y & 2 \\
 & & & & ejercicios bajo la & & alumnos & \\
 & & & & supervisión del & & & \\
 & & & & profesor. & & & \\
\hline
17/11 & & 2, 3 y 4 & Módulo 7: & Primera parte & Formativa & Profesor y & 2 \\
 & & & Polinomios. Definición, & teórica. Segunda parte & & alumnos & \\
 & & & operaciones aritméticas, & resolución de ejercicio & & & \\
 & & & propiedades, grado de polinomio. & en grupo sobre & & & \\
 & & & & polinomios supervisados & & & \\
 & & & & por el profesor. & & & \\
 & & & & & & & \\
\hline
S14:21/11 & & 2, 3 y 4 & Función racional, función & Primera parte & Formativa & Profesor y & 2 \\
 & & & racional impropia, existencia y & teórica. Segunda parte & & alumnos & \\
 & & & unicidad del cociente y resto, & los estudiantes & & & \\
 & & & algoritmo de división sintética. & resuelven ejercicios en & & & \\
 & & & & grupo sobre división & & & \\
 & & & & de polinomios & & & \\
 & & & & & & & \\
\hline
21/11 & & 2, 3 y 4 & Ayudantía. & Los estudiantes & Formativa & Ayudante y & 1 \\
 & & & & resuelven problemas del & & alumnos & \\
 & & & & Listado 7 bajo la & & & \\
 & & & & supervisión del & & & \\
 & & & & ayudante. & & & \\
\hline
22/11 & & 2, 3 y 4 & Algoritmo de Ruffini, Teorema & Primera parte & Formativa & Profesor y & 2 \\
 & & & del resto. & teórica. Segunda parte & & alumnos & \\
 & & & & los estudiantes & & & \\
 & & & & resuelven ejercicios en & & & \\
 & & & & grupo bajo la & & & \\
 & & & & supervisión del & & & \\
 & & & & profesor sobre & & & \\
 & & & & de polinomios. & & & \\
 & & & & & & & \\
\hline
24/11 & & 2, 3 y 4 & Ayudantía. & Alumnos resuelven & Formativa & Ayudante y & 1 \\
 & & & & problemas del Listado & & alumnos & \\
 & & & & 8 supervisados por el & & & \\
 & & & & ayudante. & & & \\
\hline
S15:28/11 & & 2, 3 y 4 & Raíces de polinomios y su & Primera parte & Formativa & Profesor y & 2 \\
 & & & multiplicidad. Teorema & teórica. Segunda parte & & alumnos & \\
 & & & fundamental del & los estudiantes & & & \\
 & & & álgebra. Descomposición en & resuelven ejercicios en & & & \\
 & & & factores irreducibles. & grupo sobre & & & \\
 & & & & factorización de & & & \\
 & & & & polinomios. & & & \\
 & & & & & & & \\
\hline
28/11 & & 2, 3 y 4 & Ayudantía. & Los estudiantes & Formativa & Ayudante y & 1 \\
 & & & & resuelven problemas & & alumnos & \\
 & & & & del Listado 7 bajo la & & & \\
 & & & & supervisión del & & & \\
 & & & & ayudante. & & & \\
\hline
29/11 & & 2, 3 y 4 & Descomposición en suma de & Primera parte & Formativa & Profesor y & 2 \\
 & & & fracciones parciales. & teórica. Segunda parte & & alumnos & \\
 & & & & los estudiantes & & & \\
 & & & & resuelven & & & \\
 & & & & ejercicios sobre & & & \\
 & & & & descomposición de & & & \\
 & & & & fracciones parciales. & & & \\
 & & & & & & & \\
\hline
1/2 & & 2, 3 y 4 & Ayudantía. & Alumnos resuelven & Formativa & Ayudante y & 1 \\
 & & & & ejercicios para & & alumnos & \\
 & & & & preparación para el & & & \\
 & & & & Certamen 3 bajo la & & & \\
 & & & & supervisión del & & & \\
 & & & & ayudante. & & & \\
\hline
S16:5/12 & & 2, 3 y 4 & Certamen 3 & & Sumativa & Profesor y & 2 \\
 & & & & & & alumnos & \\
\hline
22/11 & & 2, 3 y 4 & Evaluación de Recuperación & & Sumativa & Profesor y & 2 \\
 & & & & & & alumnos & \\
\hline
\end{longtable}
\end{center}


}


\section{Datos de Contacto}
\label{sec:orgceb8c24}

\begin{itemize}
\item Nombre del docente: Ramiro Rebolledo.

e-mail: ramirorebolledo@udec.cl

Oficina: 140, Departamento de Agroindustrias, Facultad de Ingeniería Agrícola.  

Horario de consultas: Viernes 11:00-12:00 hrs

\item Nombre del ayudante: Mauricio Bahamondes.

e-mail: mlagos2017@udec.cl
\end{itemize}


\section{Requisitos de la Asignatura}
\label{sec:org711556d}

Clases teóricas: Lunes 11:15-13:00 hrs y Martes 14:15-16:00 en la Sala 5 del Edificio Central.

Clases prácticas: Jueves 10:15-12:00 hrs en la Sala 5 del Edificio Central.

Ayudantías: Lunes 14:00-15:00 hrs en la Sala FIA-1.

Nota: Todas las actividades del curso son presenciales, pero de no ser posible la presencialidad en el campus se evaluará reemplazar todas las actividades presenciales por virtuales, para lo cual se utilizará Teams y Canvas. Es responsabilidad de cada estudiante asegurarse que tiene acceso a estas herramientas  en el caso de ser necesario.


\section{Recursos de Aprendizaje}
\label{sec:org0f70fec}
Los recursos de aprendizaje son los libros de la bibliografía, los cuales están disponibles en biblioteca, y los archivos con materia, listados de ejercicios propuestos, y ejercicios resueltos (pautas de evaluaciones del semestre anterior) disponibles en Canvas. Cualquier cambio en este syllabus (por ejemplo, cambio de horario) será avisada por e-mail. Es respondabilidad de cada estudiante revisar su correo periódicamente.
\end{document}
